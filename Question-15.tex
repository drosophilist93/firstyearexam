% Options for packages loaded elsewhere
\PassOptionsToPackage{unicode}{hyperref}
\PassOptionsToPackage{hyphens}{url}
%
\documentclass[
]{article}
\title{Question 15}
\author{Chloe J. Welch}
\date{6/4/2022}

\usepackage{amsmath,amssymb}
\usepackage{lmodern}
\usepackage{iftex}
\ifPDFTeX
  \usepackage[T1]{fontenc}
  \usepackage[utf8]{inputenc}
  \usepackage{textcomp} % provide euro and other symbols
\else % if luatex or xetex
  \usepackage{unicode-math}
  \defaultfontfeatures{Scale=MatchLowercase}
  \defaultfontfeatures[\rmfamily]{Ligatures=TeX,Scale=1}
\fi
% Use upquote if available, for straight quotes in verbatim environments
\IfFileExists{upquote.sty}{\usepackage{upquote}}{}
\IfFileExists{microtype.sty}{% use microtype if available
  \usepackage[]{microtype}
  \UseMicrotypeSet[protrusion]{basicmath} % disable protrusion for tt fonts
}{}
\makeatletter
\@ifundefined{KOMAClassName}{% if non-KOMA class
  \IfFileExists{parskip.sty}{%
    \usepackage{parskip}
  }{% else
    \setlength{\parindent}{0pt}
    \setlength{\parskip}{6pt plus 2pt minus 1pt}}
}{% if KOMA class
  \KOMAoptions{parskip=half}}
\makeatother
\usepackage{xcolor}
\IfFileExists{xurl.sty}{\usepackage{xurl}}{} % add URL line breaks if available
\IfFileExists{bookmark.sty}{\usepackage{bookmark}}{\usepackage{hyperref}}
\hypersetup{
  pdftitle={Question 15},
  pdfauthor={Chloe J. Welch},
  hidelinks,
  pdfcreator={LaTeX via pandoc}}
\urlstyle{same} % disable monospaced font for URLs
\usepackage[margin=1in]{geometry}
\usepackage{color}
\usepackage{fancyvrb}
\newcommand{\VerbBar}{|}
\newcommand{\VERB}{\Verb[commandchars=\\\{\}]}
\DefineVerbatimEnvironment{Highlighting}{Verbatim}{commandchars=\\\{\}}
% Add ',fontsize=\small' for more characters per line
\usepackage{framed}
\definecolor{shadecolor}{RGB}{248,248,248}
\newenvironment{Shaded}{\begin{snugshade}}{\end{snugshade}}
\newcommand{\AlertTok}[1]{\textcolor[rgb]{0.94,0.16,0.16}{#1}}
\newcommand{\AnnotationTok}[1]{\textcolor[rgb]{0.56,0.35,0.01}{\textbf{\textit{#1}}}}
\newcommand{\AttributeTok}[1]{\textcolor[rgb]{0.77,0.63,0.00}{#1}}
\newcommand{\BaseNTok}[1]{\textcolor[rgb]{0.00,0.00,0.81}{#1}}
\newcommand{\BuiltInTok}[1]{#1}
\newcommand{\CharTok}[1]{\textcolor[rgb]{0.31,0.60,0.02}{#1}}
\newcommand{\CommentTok}[1]{\textcolor[rgb]{0.56,0.35,0.01}{\textit{#1}}}
\newcommand{\CommentVarTok}[1]{\textcolor[rgb]{0.56,0.35,0.01}{\textbf{\textit{#1}}}}
\newcommand{\ConstantTok}[1]{\textcolor[rgb]{0.00,0.00,0.00}{#1}}
\newcommand{\ControlFlowTok}[1]{\textcolor[rgb]{0.13,0.29,0.53}{\textbf{#1}}}
\newcommand{\DataTypeTok}[1]{\textcolor[rgb]{0.13,0.29,0.53}{#1}}
\newcommand{\DecValTok}[1]{\textcolor[rgb]{0.00,0.00,0.81}{#1}}
\newcommand{\DocumentationTok}[1]{\textcolor[rgb]{0.56,0.35,0.01}{\textbf{\textit{#1}}}}
\newcommand{\ErrorTok}[1]{\textcolor[rgb]{0.64,0.00,0.00}{\textbf{#1}}}
\newcommand{\ExtensionTok}[1]{#1}
\newcommand{\FloatTok}[1]{\textcolor[rgb]{0.00,0.00,0.81}{#1}}
\newcommand{\FunctionTok}[1]{\textcolor[rgb]{0.00,0.00,0.00}{#1}}
\newcommand{\ImportTok}[1]{#1}
\newcommand{\InformationTok}[1]{\textcolor[rgb]{0.56,0.35,0.01}{\textbf{\textit{#1}}}}
\newcommand{\KeywordTok}[1]{\textcolor[rgb]{0.13,0.29,0.53}{\textbf{#1}}}
\newcommand{\NormalTok}[1]{#1}
\newcommand{\OperatorTok}[1]{\textcolor[rgb]{0.81,0.36,0.00}{\textbf{#1}}}
\newcommand{\OtherTok}[1]{\textcolor[rgb]{0.56,0.35,0.01}{#1}}
\newcommand{\PreprocessorTok}[1]{\textcolor[rgb]{0.56,0.35,0.01}{\textit{#1}}}
\newcommand{\RegionMarkerTok}[1]{#1}
\newcommand{\SpecialCharTok}[1]{\textcolor[rgb]{0.00,0.00,0.00}{#1}}
\newcommand{\SpecialStringTok}[1]{\textcolor[rgb]{0.31,0.60,0.02}{#1}}
\newcommand{\StringTok}[1]{\textcolor[rgb]{0.31,0.60,0.02}{#1}}
\newcommand{\VariableTok}[1]{\textcolor[rgb]{0.00,0.00,0.00}{#1}}
\newcommand{\VerbatimStringTok}[1]{\textcolor[rgb]{0.31,0.60,0.02}{#1}}
\newcommand{\WarningTok}[1]{\textcolor[rgb]{0.56,0.35,0.01}{\textbf{\textit{#1}}}}
\usepackage{longtable,booktabs,array}
\usepackage{calc} % for calculating minipage widths
% Correct order of tables after \paragraph or \subparagraph
\usepackage{etoolbox}
\makeatletter
\patchcmd\longtable{\par}{\if@noskipsec\mbox{}\fi\par}{}{}
\makeatother
% Allow footnotes in longtable head/foot
\IfFileExists{footnotehyper.sty}{\usepackage{footnotehyper}}{\usepackage{footnote}}
\makesavenoteenv{longtable}
\usepackage{graphicx}
\makeatletter
\def\maxwidth{\ifdim\Gin@nat@width>\linewidth\linewidth\else\Gin@nat@width\fi}
\def\maxheight{\ifdim\Gin@nat@height>\textheight\textheight\else\Gin@nat@height\fi}
\makeatother
% Scale images if necessary, so that they will not overflow the page
% margins by default, and it is still possible to overwrite the defaults
% using explicit options in \includegraphics[width, height, ...]{}
\setkeys{Gin}{width=\maxwidth,height=\maxheight,keepaspectratio}
% Set default figure placement to htbp
\makeatletter
\def\fps@figure{htbp}
\makeatother
\setlength{\emergencystretch}{3em} % prevent overfull lines
\providecommand{\tightlist}{%
  \setlength{\itemsep}{0pt}\setlength{\parskip}{0pt}}
\setcounter{secnumdepth}{-\maxdimen} % remove section numbering
\ifLuaTeX
  \usepackage{selnolig}  % disable illegal ligatures
\fi

\begin{document}
\maketitle

\textbf{Q15}. {[}10pt{]} Obtain the most recently dated COVID-19 Variant
Data from the California Health and Human Services (CHHS) open data
site:

\url{https://data.chhs.ca.gov/dataset/covid-19-variant-data}

Upload to gradescope a PDF format report generated from an Rmarkdown
document that demonstrates reading the above CSV file and generating the
below visualization of this data.

NB. You can chose how to make this plot and whether you want to make
improvements or stylistic changes. However, you are strongly encouraged
to use the ggplot2, lubridate and dplyr packages for this task. Please
make sure your name and PID number is on the first page and that your
report contains all of your code, text description/narrative text of why
you doing a particular task/code chunk and the resulting figure.

\hypertarget{we-will-begin-by-loading-the-necessary-packages-to-analyze-this-data-set.-these-were-already-installed-in-bggn-213-during-the-fall-quarter.}{%
\section{We will begin by loading the necessary packages to analyze this
data set. These were already installed in BGGN 213 during the fall
quarter.}\label{we-will-begin-by-loading-the-necessary-packages-to-analyze-this-data-set.-these-were-already-installed-in-bggn-213-during-the-fall-quarter.}}

\begin{Shaded}
\begin{Highlighting}[]
\FunctionTok{library}\NormalTok{(ggplot2)}
\FunctionTok{library}\NormalTok{(lubridate)}
\end{Highlighting}
\end{Shaded}

\begin{verbatim}
## 
## Attaching package: 'lubridate'
\end{verbatim}

\begin{verbatim}
## The following objects are masked from 'package:base':
## 
##     date, intersect, setdiff, union
\end{verbatim}

\begin{Shaded}
\begin{Highlighting}[]
\FunctionTok{library}\NormalTok{(dplyr)}
\end{Highlighting}
\end{Shaded}

\begin{verbatim}
## 
## Attaching package: 'dplyr'
\end{verbatim}

\begin{verbatim}
## The following objects are masked from 'package:stats':
## 
##     filter, lag
\end{verbatim}

\begin{verbatim}
## The following objects are masked from 'package:base':
## 
##     intersect, setdiff, setequal, union
\end{verbatim}

Next, we will import the COVID-19 variant data.

\begin{Shaded}
\begin{Highlighting}[]
\NormalTok{covid.data }\OtherTok{\textless{}{-}} \FunctionTok{read.csv}\NormalTok{(}\StringTok{"covid19\_variants.csv"}\NormalTok{)}
\FunctionTok{head}\NormalTok{(covid.data)}
\end{Highlighting}
\end{Shaded}

\begin{verbatim}
##         date       area area_type variant_name specimens percentage
## 1 2021-01-01 California     State        Alpha         1       1.69
## 2 2021-01-01 California     State           Mu         0       0.00
## 3 2021-01-01 California     State        Other        29      49.15
## 4 2021-01-01 California     State        Delta         0       0.00
## 5 2021-01-01 California     State         Beta         0       0.00
## 6 2021-01-01 California     State        Total        59     100.00
##   specimens_7d_avg percentage_7d_avg
## 1               NA                NA
## 2               NA                NA
## 3               NA                NA
## 4               NA                NA
## 5               NA                NA
## 6               NA                NA
\end{verbatim}

Let's call the \texttt{skim()} function from the skimr package to get a
quick overview of this dataset:

\begin{Shaded}
\begin{Highlighting}[]
\FunctionTok{library}\NormalTok{(skimr)}
\NormalTok{skimr}\SpecialCharTok{::}\FunctionTok{skim}\NormalTok{(covid.data)}
\end{Highlighting}
\end{Shaded}

\begin{longtable}[]{@{}ll@{}}
\caption{Data summary}\tabularnewline
\toprule
\endhead
Name & covid.data \\
Number of rows & 4920 \\
Number of columns & 8 \\
\_\_\_\_\_\_\_\_\_\_\_\_\_\_\_\_\_\_\_\_\_\_\_ & \\
Column type frequency: & \\
character & 4 \\
numeric & 4 \\
\_\_\_\_\_\_\_\_\_\_\_\_\_\_\_\_\_\_\_\_\_\_\_\_ & \\
Group variables & None \\
\bottomrule
\end{longtable}

\textbf{Variable type: character}

\begin{longtable}[]{@{}lrrrrrrr@{}}
\toprule
skim\_variable & n\_missing & complete\_rate & min & max & empty &
n\_unique & whitespace \\
\midrule
\endhead
date & 0 & 1 & 10 & 10 & 0 & 492 & 0 \\
area & 0 & 1 & 10 & 10 & 0 & 1 & 0 \\
area\_type & 0 & 1 & 5 & 5 & 0 & 1 & 0 \\
variant\_name & 0 & 1 & 2 & 7 & 0 & 10 & 0 \\
\bottomrule
\end{longtable}

\textbf{Variable type: numeric}

\begin{longtable}[]{@{}lrrrrrrrrrl@{}}
\toprule
skim\_variable & n\_missing & complete\_rate & mean & sd & p0 & p25 &
p50 & p75 & p100 & hist \\
\midrule
\endhead
specimens & 0 & 1.00 & 180.60 & 515.43 & 0 & 0 & 0.00 & 39.00 & 5713.00
& ▇▁▁▁▁ \\
percentage & 0 & 1.00 & 20.00 & 36.78 & 0 & 0 & 0.00 & 13.93 & 100.00 &
▇▁▁▁▂ \\
specimens\_7d\_avg & 60 & 0.99 & 182.11 & 473.34 & 0 & 0 & 0.57 & 44.93
& 3150.86 & ▇▁▁▁▁ \\
percentage\_7d\_avg & 60 & 0.99 & 20.00 & 36.77 & 0 & 0 & 0.09 & 13.22 &
100.00 & ▇▁▁▁▂ \\
\bottomrule
\end{longtable}

We can use \texttt{lubridate} to help make dealing with the dates of the
data set a little simpler.

\begin{Shaded}
\begin{Highlighting}[]
\NormalTok{covid.data}\SpecialCharTok{$}\NormalTok{date }\OtherTok{\textless{}{-}} \FunctionTok{ymd}\NormalTok{(covid.data}\SpecialCharTok{$}\NormalTok{date)}
\FunctionTok{head}\NormalTok{(covid.data}\SpecialCharTok{$}\NormalTok{date)}
\end{Highlighting}
\end{Shaded}

\begin{verbatim}
## [1] "2021-01-01" "2021-01-01" "2021-01-01" "2021-01-01" "2021-01-01"
## [6] "2021-01-01"
\end{verbatim}

Next, let's filter out the entries we do not need.

\begin{Shaded}
\begin{Highlighting}[]
\NormalTok{filtered.data }\OtherTok{\textless{}{-}} \FunctionTok{filter}\NormalTok{(covid.data, variant\_name }\SpecialCharTok{!=} \StringTok{"Total"} \SpecialCharTok{\&}\NormalTok{ variant\_name }\SpecialCharTok{!=} \StringTok{"Other"}\NormalTok{)}
\FunctionTok{head}\NormalTok{(filtered.data)}
\end{Highlighting}
\end{Shaded}

\begin{verbatim}
##         date       area area_type variant_name specimens percentage
## 1 2021-01-01 California     State        Alpha         1       1.69
## 2 2021-01-01 California     State           Mu         0       0.00
## 3 2021-01-01 California     State        Delta         0       0.00
## 4 2021-01-01 California     State         Beta         0       0.00
## 5 2021-01-01 California     State        Gamma         0       0.00
## 6 2021-01-01 California     State      Epsilon        28      47.46
##   specimens_7d_avg percentage_7d_avg
## 1               NA                NA
## 2               NA                NA
## 3               NA                NA
## 4               NA                NA
## 5               NA                NA
## 6               NA                NA
\end{verbatim}

Let's use \texttt{ggplot2} to plot this data. The title of the plot will
be ``COVID-19 Variants in California'' with the dates on the x-axis and
the percentage of sequenced specimens on the y-axis. The date breaks are
equal to 1 month so each individual month is clearly represented. The
different colors represent the different variants of COVID-19 and are
denoted in the legend to the right of the plot.

\begin{Shaded}
\begin{Highlighting}[]
\NormalTok{covid.data.plot }\OtherTok{\textless{}{-}} \FunctionTok{ggplot}\NormalTok{(filtered.data, }\FunctionTok{aes}\NormalTok{(date, percentage)) }\SpecialCharTok{+}
  \FunctionTok{geom\_line}\NormalTok{(}\FunctionTok{aes}\NormalTok{(}\AttributeTok{color =}\NormalTok{ variant\_name)) }\SpecialCharTok{+}
  \FunctionTok{labs}\NormalTok{(}\AttributeTok{x=}\StringTok{""}\NormalTok{, }\AttributeTok{y=}\StringTok{"Percentage of sequenced specimens"}\NormalTok{, }
       \AttributeTok{title=} \StringTok{"COVID{-}19 Variants in California"}\NormalTok{, }\AttributeTok{color=}\StringTok{""}\NormalTok{) }\SpecialCharTok{+} 
  \FunctionTok{scale\_x\_date}\NormalTok{(}\AttributeTok{date\_breaks=}\StringTok{"1 month"}\NormalTok{, }\AttributeTok{date\_labels=}\StringTok{"\%b \%Y"}\NormalTok{) }
  
\NormalTok{covid.data.plot }\SpecialCharTok{+} \FunctionTok{theme\_bw}\NormalTok{() }\SpecialCharTok{+} 
  \FunctionTok{theme}\NormalTok{(}\AttributeTok{axis.text.x =} \FunctionTok{element\_text}\NormalTok{(}\AttributeTok{angle =} \DecValTok{45}\NormalTok{, }\AttributeTok{hjust =} \DecValTok{1}\NormalTok{))}
\end{Highlighting}
\end{Shaded}

\includegraphics{Question-15_files/figure-latex/unnamed-chunk-6-1.pdf}

\end{document}
